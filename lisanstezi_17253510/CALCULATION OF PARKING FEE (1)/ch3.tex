\chapter{HİPOTEZ}\label{CH2}
\section{Tezin Hipotezi}
Yapılan bu uygulamada bir otoparktaki araçların görüntü işleme ile plakaları tanınarak sqlite veritabanında saklanır. Saklanılan veriler ile plakalara bağlı otopark ücret hesabı sonuçları elde edilir.\\
\cite{platereg} İlk olarak plaka tanıma işleminin nasıl gerçekleştiğine değineceğiz.\\
Plaka tanıma sisteminde kullanılan algoritma aşaması 3 kademe şeklindedir.\\
Bu aşamalardan ilki alınan görüntü üzerinde plaka yerinin  bulunması, ikincisi bulunan plakanın üzerindeki karakterlerin parçalanması ve son olarak elde edilen karakterlerin tanıma işlemidir. Plaka sistemleri üzerine yapılan çalışmaların genelinde hazır resim filtreleme yöntemleri kullanılarak plaka yeri tespit edilmiştir. \cite{filtreleme} Bu filtreler rgb uzayında bulunan bir görüntüyü griye çevirme ,medyan alma , bileteral filtre uygulama, histogram eşitleme, kenar bulma filtresi, eşik değeri belirleme, kontur işlemi, maskeleme işlemi gibi filtrelerdir.\\
Bu fitreler opencv kütüphanesi içinde bulunan hazır filtrelerdir. Bu proje boyunca kullanılan filtreler plaka tanıma işlemi için hazırlanmış filtreler değildir. Genel olarak görüntü işlemede kullanılan filtrelerdir.\\
\cite{video_proc} Gerçek zamanlı bir plaka tanıma sisteminde kameradan alınan görüntüde ilk amaç hızlı bir şekilde plaka yerini tespit etmektir. Bunun için tüm görünütüyü genel olarak kullanılan opencv filtrelerinden geçirmek, ve görüntü üzerinde komşu pikselleri incelemek plaka tanıma işlemi için uzun süre almaktadır. Plaka tanıma sisteminde ana amaç plaka bölgesini tespit etmek olduğu için tüm görüntü üzerinde işlem yapmak süreyi uzatmaktadır.\\
Çalışma yapılırken filtreler uygulandığında önceki filtrelerle kontrol ederek görüntü üzerinde bozulma olup olmadığı kontrolleri yapılmıştır. Görüntü üzerinde piksel çıkarma işlemleri yapılarak bozulmalar kontrol edilmiştir.Bu sayade plaka yerinin tespiti işleminin hatasız bir biçimde yapılması için gerekli önlemler alınmıştır. \cite{image_proc} Plaka bölgesinin bulunması sırasında aracın üzerinde bulunan plakaya benzeyen marka yazısı, logo veya buna benzer karakterlerin olması ile bu yerlerin plaka olarak belirlenmesi hatası ile bazen karşılaşılmıştır.Gece alınan görüntü üzerinde çalışma işlemi gerçekleştirilmemiştir. Plaka tanıma işleminde kullanılan görüntülerin çözünürlük oranı arttıkça plaka bölgesini bulma işlemindeki başarı oranı doğrusal olarak artış göstermiştir. Bu çalışmada farklı renklerdeki araçlar üzerinde denenmiştir. Araçların rengi farketmeksizin plaka bölgesini belirleme işlemi başarıyla gerekleşmiştir. Araç üzerindeki görüntüler hem ön tarfata hem de arka taraftan alınan görüntülerde denenmiştir.\\
Her iki açıdan da alınan görüntülerde de başarı oranı neredeyse aynıdır. \cite{detection} Karakter tanıma işlemlerinde şablon eşleştirme ve yapay sinir ağları bulunmaktadır. Şablon eşleştirme yönteminde adından anlaşılacağı gibi önceden bir şablon oluşturmak gerekir. Elimizde bulunan şablonla plaka üzerinde plaka üzerinde kontrol işlemi gerçekleştirilir. Şablon eşleştirme yönteminin dezavantajı eğer plaka üzerindeki bir karakter hafif bir bozulma yaşamışsa şablon eşleştirme metodu ile karakterleri tanıma işlemi başarısız olacaktır. \\
\cite{segmentation} Şablon eşleştirme yöntemi plaka tanıma sistemi gibi yüksek derecede doğruluk bilgisi gereken yerlerde kullanılması sorun yaşatabilir. Şablon eşleştirme yerine yapay sinir aları yöntemi kullanarak daha yüksek oranda başarı oranına ulaşılabilir. Sonuç olarak alışveriş merkezlerinde, sınır kapılarında, üniversite kampüsüne giriş çıkışlarda, otoparklarda, emniyet müdürlüğü gibi alanlarda aracın giriş, çıkış, çalıntı veya istatiksel gibi verilerin tutulması gereken yerlerde kullanılmaktadır.Plaka tanıma sistemi günümüzde neredeyse zorunlu hale gelmiştir. Plaka tanıma sistemi sayesinde güvenlik zafiyetleri engellenmesi için yardım etmekte ve insan gücünü en aza indirmektedir. İnsanların 24 saat boyunca kağıtta kimin girip çıktığını not tutması ihtiyacı ortadan kalkmıştır. Tüm önlemler alınarak yapılan bir plaka tanıma sistemi sayesinde 24 saat boyunca sürekli aynı verimde çalışabilir. Plaka tanıma sisteminin gelecekte aracın bulunduğu her noktada hayatımızda olacağı düşünülmektedir. \\
Plaka tanıma işlemi sonucunda araçların plakaların ,kullanıcıdan girilen konum bilgileri ve otoparka giriş saatleri SQLite veritabanında kayıt olarak tutulur. Bu kayıtlar otoparktan araç çıkarken , python ile çekilir (select edilir) ve nihai olarak istediğimiz otopark ücreti ve plakaya sahip aracın konumunun bize döndürülmesine olasılık sağlar.


