%%%%%%%%%%%%%%%%%%%%%%%%%%%%%%%%%%%%%%%%%%%%%%%%%%%%%%%%%%%%%%%%%
\chapter{ATIFLAR, ALINTILAR ve DİPNOTLAR}\label{ch:ifnecch4}
%%%%%%%%%%%%%%%%%%%%%%%%%%%%%%%%%%%%%%%%%%%%%%%%%%%%%%%%%%%%%%%%%
Bu bölümde atıflar, alıntılar ve dipnotların nasıl olması gerektiği hakkında bilgi verilecektir.
\section{Atıflar (kaynakların metin içinde gösterimi)}
%%%%%%%%%%%%%%%%%%%%%%%%%%%%%%%%%%%%%%%%%%%%
%Kaynak gösterminde yazar soyadına göre ve numara ile atıf verme 
%yöntemlerinden biri tercih edilir ve tüm tezde aynı yöntem  kullanılır. 
%Numara ve soyad ile göstermin ikisi beraber kullanılmaz.
%%%%%%%%%%%%%%%%%%%%%%%%%%%%%%%%%%%%%%%%%%%%


\subsection{Numara ile atıf verme}

Metin içinde [ ] köşeli parantez içinde numaralandırılır. Tezde ilk verilen kaynak [1] numara ile başlar ve veriliş sırasına göre numaralandırılır.

Kaynaklara metin içerisinde aşağıdaki biçimlerde atıf yapılır.
\cite{Wegener2000629, Zuckerman199486, Wolchik2000843, Burke74, harper2007, unesco, mccaffrey88, moore91}

\cite{Wegener2000629}	1 nolu kaynak,

\cite{Wegener2000629, Zuckerman199486, Wolchik2000843}	1 ve 3 arası (1, 2 ve 3 nolu ) kaynaklar,

\cite{Wegener2000629, Wolchik2000843}	1 ve 3 nolu kaynaklar, 

\cite{Wegener2000629, Wolchik2000843, moore91} 1, 3 ve 8 nolu kaynaklar,

\cite{Wegener2000629, Wolchik2000843, Burke74, harper2007, unesco, mccaffrey88, moore91}	1 ve 3 ile 8 nolu kaynaklar arasındaki kaynaklar.

Web sayfalarına yapılan atıflar \cite{WinNT}. 

Aynı isimli birden fazla cildi olan kaynakların, kullanılan her bir cildine ayrı kaynak numarası verilmelidir. 

\section{Alıntılar}

Genel olarak alıntılar kelime, imla ve noktalama bakımından aslına uygun olarak yapılır. Alıntı yapılan parçada bir yanlış varsa, doğrusu köşeli parantez içerisinde belirtilmek koşuluyla metin aynen nakledilir.

Kırk kelimeden daha az uzunluktaki kısa alıntılar çift tırnak içerisinde verilir. Alıntının sonunda ilgili kaynağa atıf yapılıp atıftan sonra nokta koyulur. 

Kırk kelimeden fazla olan uzun alıntılar tırnak içerisinde gösterilmezler. Uzun alıntılar soldan 1 sekme (1,27 cm) içerden verilir. İçerden verilen uzun alıntılarda, 2 yazı karakteri daha küçük karakter kullanılır. Ancak, çok sık ve çok uzun alıntılardan kaçınılması tavsiye edilir. Kısa alıntılardan farklı olarak noktalama atıftan sonra değil de önce yapılır. Örneğin;.(s. 196)gibi.

40 kelimeden fazla olan alıntı örneği;

Ana metin ana metin ana metin ana metin ana metin ana metin ana metin ana metin ana metin ana metin Others have contradicted this view:

Co-presence does not ensure intimate interaction among all group members.Consider large-scale social gatherings in which hundreds or thousands of people gather in a location to perform a ritual or celebrate an event. In these instances, participats are able to see the visible manifestation of the group, the physicsl gathering, yet their ability to make direct, intimate  connections with those around them is limited by the sheer magnitude of the assembly (Purcell, 1997, ss. 111-112).

Devam eden metin devam eden metin devam eden metin devam eden metin devam eden metin devam eden metin devam eden metin devam eden metin devam eden metin.

Cümle başındaki alıntı örnekleri;

According to Jones (1998), "Students often had difficulty using APA style,especially when it was their first time" (s. 199).

“Critser (2003) noted that despite growing numbers of overweight Americans, many health care providers still “remain either in ignorance or outright denial about the health danger to the poor and the young” (s. 5).

Critser (2003) noted that despite growing numbers of overweight Americans, many health care providers still “remain either in ignorance or outright denial about the health danger to the poor and the young” (Critser, 2003, s. 5).

Cümle arasındaki kısa alıntı örneği;

Interpreting these results, Robbins et al. (2003) suggested that the “therapists in dropout cases may have inadvertently validated parental negativity about the adolescent without adequately responding to the adolescent’s needs or concerns” (s. 541)contributing to an overall climate of negativity.

Cümle sonundaki kısa alıntı örneği;

%Confusing this issue is the overlapping nature of roles in palliative care, whereby ``medical needs are met by those in the medical disciplines; nonmedical needs may be addressed by anyone on the team'' (Csikai & Chaitin, 2006, s. 112).

Alıntılar hakkında detaylı bilgiler enstitülerin internet sitelerinden ve ilgili bağlantılardan bulunabilir.

\section{Dipnotlar}

Tezlerde içeriği genişletici, güçlendirici veya ilave nitelikteki bilgiler (içerik dipnotu) kullanılabilir. \footnote{Dipnotlar ile kaynak gösterimi yapılmaz. Dipnotlar tez içerisinde içeriği genişletici, güçlendirci veya ilave nitelikteki bilgileri vermek için kullanılır. Verilen genişletici, güçlendirci veya ilave nitelikteki bilgiler zorunlulukla kaynak içeriyorsa bu kaynak mutlaka kaynaklar bölümünde verilmelidir.}

Dipnot numaraları alıntının hemen sonuna koyulur. Alıntı paragrafsa dipnot numarası paragrafın son kelimesinin üzerine, alıntı bir kavram veya isimse, bu defa kavram veya ismin hemen üzerine yazılır. 

Metin içerisindeki dipnot numarası; satır hizasının üzerinde \footnote{Dipnot, ilgili sayfanın altına metinden 2 karakter küçük yazı ile yazılmalıdır} şeklinde görünür olmalıdır. Numara sonrasında herhangi bir noktalama işareti konmamalıdır.

Dipnot, ilgili sayfanın altına metinden 2 karakter küçük yazı ile yazılmalıdır. 

Dipnot çizgisi ile dipnot numarası arasında bir aralık; dipnot numarası ile dipnotun ilk satırı arasında ise yarım aralık bırakılmalıdır. Dipnotlar metinden ince yatay bir çizgi ile ayrılmalıdır.

Dipnotlarla ilgili ayrıntılı bilgiler enstitülerin internet sitelerinden ve ilgili bağlantılardan bulunabilir

\section{İkinci Derece Başlık Nasıl: İlk Harfler Büyük}

başlıklar...

başlıklar... 

\subsection{Üçüncü derece başlık nasıl: ilk harf büyük diğerleri küçük}

başlıklar...

\subsubsection{Dördüncü derece başlık nasıl: ilk harf büyük diğerleri küçük}

başlıklar...

{\bf{Beşinci derece başlık: dördüncü dereceden sonrası numaralandırılmaz}}

şekil...

\begin{figure}[h!]
 \centering
 \includegraphics[width=230pt,keepaspectratio=true]{./fig/sekil6}
 % sekil3.eps: 0x0 pixel, 300dpi, 0.00x0.00 cm, bb=14 14 1155 740
 \vspace*{2mm}
 \caption{Örnek şekil.}
 \label{fig:4-1}
\end{figure}

çizgele...

\begin{table*}[h!]
{\setlength{\tabcolsep}{14pt}
\caption{Çizelge örneği.}
\begin{center}
\vspace{-6mm}
\begin{tabular}{cccc}
\hline\hline
Kolon A & Kolon B & Kolon C & Kolon D \\
\hline
Sat\i r A & Sat\i r A & Sat\i r A & Sat\i r A \\
Sat\i r B & Sat\i r B & Sat\i r B & Sat\i r B \\
Sat\i r C & Sat\i r C & Sat\i r C & Sat\i r C \\
\hline
\end{tabular}
\vspace{-6mm}
\end{center}
\label{tableforCh4-2}}
\end{table*}

Çizelge... 

